\documentclass[12pt]{article}
\usepackage{letltxmacro}
\usepackage{makeidx}
\usepackage{multirow}
\usepackage{multicol}
\usepackage[dvipsnames,svgnames,table]{xcolor}
\usepackage{graphicx}
\usepackage{epstopdf}
\usepackage{ulem}
\usepackage{hyperref}
\usepackage{amsmath}
\usepackage{amssymb}
\title{}
\usepackage[paperwidth=595pt,paperheight=841pt,top=72pt,right=72pt,bottom=72pt,left=72pt]{geometry}

\makeatletter
	\newenvironment{indentation}[3]%
	{\par\setlength{\parindent}{#3}
	\setlength{\leftmargin}{#1}       \setlength{\rightmargin}{#1}%
	\advance\linewidth -\leftmargin       \advance\linewidth -\rightmargin%
	\advance\@totalleftmargin\leftmargin  \@setpar{{\@@par}}%
	\parshape 1\@totalleftmargin \linewidth\ignorespaces}{\par}%
	
\makeatother 

% new LaTeX commands


\begin{document}

\pagenumbering{gobble}
\vspace{4cm}
\begin{center}
\begin{indentation}{0pt}{0pt}{0pt}
\textbf{{\Large EFFICIENT BANDWIDTH UTILISATION IN WIRELESS MESH NETWORKS}}
\end{indentation}
\end{center}

\vspace{0.5cm}

\begin{center}
\begin{indentation}{0pt}{0pt}{0pt}
{\normalsize{\textbf {\textit {Submitted by}}}}
\end{indentation}
\end{center}
\vspace{0.5cm}

\begin{center}
\begin{indentation}{0pt}{0pt}{0pt}
{\normalsize \bf ASHISH GUPTA}
\end{indentation}
\end{center}

\begin{center}
\begin{indentation}{0pt}{0pt}{0pt}
{\normalsize \bf CHANDRIKA PARIMOO}
\end{indentation}
\end{center}

\begin{center}
\begin{indentation}{0pt}{0pt}{0pt}
{\normalsize \bf RUTUJA SHAH}
\end{indentation}
\end{center}

\begin{center}
\begin{indentation}{0pt}{0pt}{0pt}
{\normalsize \bf SUDIPTO CHATTERJEE}
\end{indentation}
\end{center}
\vspace{1cm}

\begin{center}
\begin{indentation}{0pt}{0pt}{0pt}
{\normalsize \bf Pune Institute of Computer Technology}
\end{indentation}
\end{center}

\begin{center}
\begin{indentation}{0pt}{0pt}{0pt}
{\normalsize \bf Team ID: BQOS179}
\end{indentation}
\end{center}

\pagebreak

\begin{indentation}{0pt}{0pt}{0pt}
\textbf{{{\Large Introduction}}}
\end{indentation}
\vspace{0.5cm}
\begin{indentation}{0pt}{0pt}{0pt}
{\normalsize \hspace{1cm} Wireless Mesh Networks(WMN) have been emerging in the last couple of years as a cost effective alternative to traditional wired access networks. A typical WMN combines a fixed network (backbone) and a mobile network (backhaul). The nodes in a WMN often act as relays, forwarding traffic to or from other mesh nodes, or providing localized connectivity to mobile or pervasive wireless devices, such as laptops, desktops and other mobile clients. Some of these nodes, called mesh gates act as gateways that are directly connected to the internet. Among the various variants of mesh networks, IEEE 802.11s is the IEEE standard for Wireless Mesh Networks and is also present in the Linux kernel. This project uses o11s (open80211s), which is the implementation of IEEE 802.11s in the Linux kernel, as the underlying mesh network.}
\end{indentation}

\begin{indentation}{0pt}{0pt}{0pt}
{\normalsize \hspace{1cm}Currently every mesh node is associated with one of the mesh gates. This type of single association involves entire traffic being directed towards the mesh gate, which has a fixed bandwidth that has to be shared by all its clients, the throughput per client is reduced as the number of client nodes associated with that node increases. In addition, when several clients associate with a single mesh gate in obtaining a path towards the gateway, the available client communication bandwidth has to be shared by all the clients as well. These are additional constraints that reduce the effective link capacity due to the multipoint-to-point nature of the traffic.}
\end{indentation}
\begin{indentation}{0pt}{0pt}{0pt}
{\normalsize \hspace{1cm}This project aims to improve the utilization of available bandwidth by using multiple internet access gateways inside a Wireless Mesh Network instead of the usual association with a single external gateway. This will hence provide a transparent solution so that pre-existing applications can run on it without changes. Traffic sensitive load sharing is also incorporated for the selection of gateways.}
\end{indentation}

\begin{indentation}{0pt}{0pt}{0pt}
{\normalsize \hspace{1cm}The benefits include increased network throughput as well as saving hardware cost due to elimination of the need for multiple radios.}
\end{indentation}

\begin{indentation}{0pt}{0pt}{0pt}
\vspace{1cm}
\textbf{{{\Large Problem Statement}}}
\end{indentation}
\vspace{0.5cm}

\begin{indentation}{0pt}{0pt}{0pt}
{\normalsize \hspace{1cm} Enhance bandwidth utilization in Wireless Mesh Networks using a decentralized approach to improve throughput and facilitate its use by protocols such as HTTP and Bittorrent.}
\end{indentation}

\begin{indentation}{0pt}{0pt}{0pt}
\vspace{1cm}
\textbf{{{\Large Solution}}}
\end{indentation}
\vspace{0.5cm}
\begin{indentation}{0pt}{0pt}{0pt}
{\normalsize \hspace{1cm} The devised solution tries to make use of all the gateways present in the wireless mesh network. Network architecture of Wireless Mesh Networks can be classified into infrastructure and hybrid wireless mesh networks. The infrastructure WMN's include mesh routers that form an infrastructure so that mesh clients can connect to them. The mesh routers in these networks are stable and static having higher link speeds. In order to efficiently use bandwidth in these network architectures we use the source port as the deciding factor and a unique mapping constraint as one port is allocated to an application for one transaction. Thus basis of mapping traffic is on a flow basis so that all the packets corresponding to a single socket are routed across the same path as the others in the same flow. This helps preserve the sanity of applications as well as not worry about deep packet inspection in order to determine which packet is meant for which application and which host thereby eliminating the need for additional book-keeping on a per packet basis.}
\end{indentation}

%\pagebreak

\begin{indentation}{0pt}{0pt}{0pt}
{\normalsize \hspace{1cm} On the other hand in hybrid WMN's the mesh network is fluid i.e. it consists of dynamic mesh routers such has phones having internet capabilities. In this type of network which comprises of both stable static routers having comparatively higher link speeds and dynamic routers(such as phone) having lower link speeds, it is necessary that we take into account their capabilities before routing the traffic via them. In order to cater to such networks the outgoing traffic from each node is routed through a set of suitable gateways chosen from the available gateways. The available gateways are categorized based on the metrics which include the bandwidth and the latency of the link. The category of gateway is then mapped to the type of request being made at the node. While the total and available bandwidth help in deciding the load on the mesh gate, the latency which also takes into account the hop count while routing to the gateway is a critical factor in routing http and VoIP traffic. Network capacity can increase linearly with the number of gateways only with proper load balancing and resource provisioning. This kind of proper association, which prevents the formation of bottlenecks and distributes the network load evenly realizes the linear capacity increase. Packets meant for the external networks are routed through one of the available gateways and therefore the next hop is set to that specific gateway. For packets meant for the internal network, the next hop is set to the destination itself as the nodes are available in the link local network and are therefore reachable without requiring the use of any gateway.}
\end{indentation}


\begin{indentation}{0pt}{0pt}{0pt}
{\normalsize \hspace{1cm} Through experimental setups using click modular router it is demonstrable that unless the underlying hardware link capacity does not become the bottleneck, linear gain in the network throughput can be seen.}
\end{indentation}

\begin{indentation}{0pt}{0pt}{0pt}
\vspace{1cm}
\textbf{{{\Large Presentation}}}
\end{indentation}
\vspace{0.5cm}
\begin{indentation}{0pt}{0pt}{0pt}
{\normalsize \hspace{1cm} In order to verify the working of this scheme, we will present it using four machines. The Client, C, has access to two other machines (the gates, G1 and G2) which in turn can access the fourth machine, the server S. This scenario effectively replicates the setup on an Internet enabled network where a client has access to two internet connections, both of which lead to a specific server somewhere on the Internet. The client does not have direct access to the server's resources as it is enforced using scripts on the client.}
\end{indentation}

\begin{indentation}{0pt}{0pt}{0pt}
{\normalsize \hspace{1cm} A fixed set of unchanging files will be downloaded from the server to the client using three path combinations. The first being only via G1, C - G1 - S. The second path used was only via G2, C - G2 - S. The third path used both the paths, C - G1, G2 - S,  through the gates to fetch those files. Comparing the time required to download the files via the above three paths will prove the aim of our project.}
\end{indentation}

\begin{indentation}{0pt}{0pt}{0pt}
{\normalsize \hspace{1cm} Another demonstration includes using two connections at the same time, a bittorrent client will be able to achieve higher download speeds compared to when using a single connection. This proves that this solution is applicable to real networks and will help the users exploit all available bandwidth on the network which would otherwise be left unused.}
\end{indentation}

\pagebreak

\begin{indentation}{0pt}{0pt}{0pt}
\textbf{{{\Large References}}}
\end{indentation}

\vspace{1.5cm}

\begin{indentation}{0pt}{0pt}{0pt}
[1] I. F. Akyildiz, X. Wang, and W. Wang, “Wireless mesh networks: a survey,” Computer Network. ISDN Syst., vol. 47, pp.
\end{indentation}

\vspace{1cm}

\begin{indentation}{0pt}{0pt}{0pt}
[2] Mihail L. Sichitiu Dept. of  Electrical and Computer Engineering, North Carolina State 
University, “Wireless Mesh Networks : Opportunities and Challenges”.
\end{indentation}

\vspace{1cm} 

\begin{indentation}{0pt}{0pt}{0pt}
[3] Vishal Sevani, Bhaskaran Raman, CS Dept, IIT Bombay, “Understanding HTTP traffic performance in TDMA mesh networks”.
\end{indentation}

\vspace{1cm}

\begin{indentation}{0pt}{0pt}{0pt}
[4] Franesco Paolo Déila, Giovanni Di Stasi, Stefao Stallone, Roberto Canonico , niversito di Napoli Federico, Napoli, “Bittorrent traffic optimization in Wireless Mesh Networks”.
\end{indentation}

\vspace{1cm}

\begin{indentation}{0pt}{0pt}{0pt}
[5] Hiertz, G.R.Denteneer, D.Max, S.Taori, R.Cardona, J.Berlemann, L.Walke, "IEEE 802.11s: The WLAN Mesh Standard", Wireless Communications, IEEE  (Volume:17 ,  Issue: 1 ),February 2010
\end{indentation}

\vspace{1cm}

\begin{indentation}{0pt}{0pt}{0pt}
[6] Madhusudan Singh, Song-Gon Lee2, HoonJae Lee "Non-root-based Hybrid Wireless Mesh Protocol for Wireless Mesh Networks", International Journal of Smart Home Vol. 7, No. 2, March, 2013
\end{indentation}

\vspace{1cm}

\begin{indentation}{0pt}{0pt}{0pt}
[7] Mohammed I. Gumel, Nasir Faruk and A.A. Ayeni, "ROUTING WITH LOAD BALANCING IN WIRELESS MESH NETWORKS", International Journal of Current Research Vol. 3, Issue, 7, pp.087-092, July, 2011
\end{indentation}

\vspace{1cm}

\begin{indentation}{0pt}{0pt}{0pt}
[8] open80211s [Online] Available: www.github.com/open80211s.
\end{indentation}
\end{document}



